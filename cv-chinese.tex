% 中文编译使用XeLatex
% for further details and (very nice) examples see the official
% documentation at: http://www.ctan.org/pkg/moderncv
% the placeholder picture in this example is taken from the moderncv
% package.

\documentclass[11pt,a4paper]{moderncv}
\usepackage[ngerman]{babel}
\usepackage{verbatim}
\usepackage[noindent]{ctex}



% arguments: 'blue', 'orange', 'red', 'green', 'grey'
% and 'roman' (for roman fonts, instead of sans serif fonts)
\moderncvtheme[blue]{classic}

% character encoding
\usepackage[utf8]{inputenc}
\usepackage[T1]{fontenc}

% adjust page margins
\usepackage[scale=0.8]{geometry}
\recomputelengths

% adjust the column width
\sethintscolumnlength{3cm}


% personal data
\firstname{杨}
\familyname{杰}

% optional stuff, uncomment/remove these lines if not needed
\address{南京大学仙林校区}{南京}
%\phone{+86 15365186708}
\email{467287824@qq.com}
%\photo[84pt]{picture/picture}

\mobile{15365186708}
%\title{Resume title}
%\fax{fax}
%\quote{Everything not saved will be lost. ---\textit{Nintendo Quit Screen}}

% uncomment to suppress automatic page numbering for CVs longer than one page
%\nopagenumbers{}


\begin{document}
	
	% default section title for publications is 'Literatur'. we want
	% to use 'Publikationen' instead.
	\renewcommand{\refname}{论文发表}
	\maketitle
	
	\section{个人简介}
	\cvline{}
	{
		我的兴趣主要在FPGA、嵌入式系统、计算机体系结构、异构计算等方面。
		\newline
		本科期间,项目集中在FPGA和ARM+FPGA异构系统。
		\newline
		研究生期间,我的研究课题是数据中心网络加速,主要关注RDMA拥塞控制和中间件。
	}
	\section{教育经历}
	\cventry{2015--至今}{硕士}{南京大学计算机科学与技术系}{}{}{}
	\cventry{2010--2014}{本科}{南京大学计算机科学与技术系}{}{}{}
	\section{获奖情况}
	\cvline{2012--2013}{人民奖学金三等奖}
	\cvline{2011--2012}{国家励志奖奖学金,人民奖学金一等奖}	
	\section{技能}
	\cvline{编程语言}{Verilog HDL, C, C++, Python, Java, Matlab}
	
	\section{项目经历}
	\cventry{2016--至今}{南京大学}{}{南京}{}
	{
		\smallskip
		\textbf{基于FPGA的RDMA网卡实现}
		\medskip \\
		RDMA是数据中心网络传输技术新趋势。本项目是将目前主流的RDMA协议实现在FPGA上,作为RDMA网络研究平台。项目采用Xilinx HLS结合Microblaze,在Xilinx Kintex7芯片实现,可以达到10Gbps速率发送RDMA报文。同时,该平台可以支持与商业RDMA网卡互联。\newline
	}
	\cventry{2015-3--2015-9}{南京大学}{}{南京}{}
	{
		\smallskip
		\textbf{基于FPGA的软件定义中间件实现}
		\medskip \\
		本项目中,我们定义类C伪语言实现自定义中间件的功能,最终伪语言运行在FPGA上。伪语言首先经过编译器转换为Xilinx HLS,然后转成verilog烧写在FPGA芯片上,主要实现的中间件有NAT,AES加密解密等。
	}
	\cventry{2013--2015}{南京大学}{}{南京}{}
	{
		\smallskip
		\textbf{SimMIPS: 基于FPGA的MIPS嵌入式系统}
		\medskip \\
		SimMIPS是基于FPGA实现的片上系统,包括MIPS32核心,总线以及多种外设等。MIPS32核心按照标准MIPS32指令集实现,同时包括有内存管理单元(MMU)。使用C/C++开发的应用程序或操作系统,经过MIPS交叉编译工具编译后的代码可以直接在MIPS32核心中运行。更多细节可以访问\url{https://github.com/jackyangNJ/SimMIPS}。 \newline
	}
	\cventry{2013--2014}{南京大学}{}{南京}{}
	{
		\smallskip
		\textbf{SmartCar: 室内导航智能小车}
		\medskip  \\
		SmartCar旨在结合多种传感器技术,实现室内导航的功能。我们基于智能车zrobot平台,该平台的控制器是融合ARM核心的FPGA芯片, 使用C/C++、Java 和 Verilog HDL语言,结合图像识别以及传统惯性导航技术,实现室内定位。图像识别方法是在室内张贴含有位置信息的二维码,小车在行走中可以扫描定位。项目相关源码可以访问\url{https://github.com/jackyangNJ/SmartCar}.\newline
	}
	
	\cventry{2013 暑期}{计算所}{中科院}{北京}{}
	{
		\smallskip
		\textbf{探索Xilinx MicroBlaze性能以及搭建平台}
		\medskip  \\
		本项目是辅助计算所项目研究。我的工作是在Xilinx FPGA上,测试MicroBlaze运行Linux的整数计算性能,同时搭建一个多核MicroBlaze系统原型,各核心之间使用高速AXI Stream通信。
	}
	
	
	% Publications from a BibTeX file without
	% multibib\renewcommand*{\bibliographyitemlabel}{\@biblabel{\arabic{enumiv}}}
	% for BibTeX numerical labels
	
	\nocite{*}
	% unsrt means the publications are displayed in the order in which
	% they appear in the publications.bib file
	\bibliographystyle{unsrt}
	\bibliography{Publications}
	
	% Publications from a BibTeX file using the multibib package
	%\section{Publications}
	%\nocitebook{book1,book2}
	%\bibliographystylebook{plain}
	%\bibliographybook{publications}
	%\nocitemisc{misc1,misc2,misc3}
	%\bibliographystylemisc{plain}
	%\bibliographymisc{publications}
	
	
	% remove the comment-tag to diplay a signature under your cv
	\begin{comment}
	\begin{flushleft}
	\vspace{2cm}\hspace{3.5cm}
	Musterstadt, den \today
	\end{flushleft}
	
	
	\vspace{1cm} \hspace{3.5cm}
	% you could include a scanned signature here
	%\includegraphics[scale=0.75]{}\\
	Max Mustermann
	\end{comment}
	
	
	
\end{document}
