% 中文编译使用XeLatex
% for further details and (very nice) examples see the official
% documentation at: http://www.ctan.org/pkg/moderncv
% the placeholder picture in this example is taken from the moderncv
% package.

\documentclass[11pt,a4paper]{moderncv}
\usepackage[ngerman]{babel}
\usepackage{verbatim}
\usepackage[noindent]{ctex}



% arguments: 'blue', 'orange', 'red', 'green', 'grey'
% and 'roman' (for roman fonts, instead of sans serif fonts)
\moderncvtheme[blue]{classic}

% character encoding
\usepackage[utf8]{inputenc}
\usepackage[T1]{fontenc}

% adjust page margins
\usepackage[scale=0.8]{geometry}
\recomputelengths

% adjust the column width
\sethintscolumnlength{3cm}


% personal data
\firstname{杨}
\familyname{杰}

% optional stuff, uncomment/remove these lines if not needed
\address{南京大学仙林校区}{南京}
%\phone{+86 15365186708}
\email{467287824@qq.com}
%\photo[84pt]{picture/picture}

\mobile{15365186708}
%\title{Resume title}
%\fax{fax}
%\quote{Everything not saved will be lost. ---\textit{Nintendo Quit Screen}}

% uncomment to suppress automatic page numbering for CVs longer than one page
%\nopagenumbers{}


\begin{document}
	
	% default section title for publications is 'Literatur'. we want
	% to use 'Publikationen' instead.
	\renewcommand{\refname}{Publikationen}
	\maketitle
	
	\section{教育经历}
	\cventry{2010--2014}{本科}{南京大学计算机科学与技术系}{}{}{}
	\cventry{2015--至今}{研究生}{南京大学计算机科学与技术系}{}{}{}
	\section{个人简介}
	\cvline{}
	{
		我的兴趣主要在计算机体系结构、FPGA、异构计算、嵌入式系统等方面。
		\newline
		本科期间,主要在FPGA和ARM+FPGA异构系统做过一些项目。
		\newline
		研究生期间,研究课题是RDMA网络拥塞控制,RDMA网卡上实现新的功能,加速数据中心网络。
	}
	
	\section{获奖情况}
	\cvline{2011--2012}{国家励志奖奖学金,人民奖学金一等奖}
	\cvline{2012--2013}{人民奖学金三等奖}
	
	\section{技能}
	\cvline{编程语言}{Verilog HDL, C, C++, Python, Java, Matlab}
	
	\section{项目经历}
	\cventry{2016--至今}{南京大学}{}{南京}{}
	{
		\smallskip
		\emph{基于FPGA的RDMA网卡实现}
		\medskip \\
		RDMA是数据中心网络传输技术新趋势。本项目是将目前主流的RDMA协议实现在FPGA网卡上,作为RDMA网络研究平台。项目采用Xilinx HLS结合Microblaze,在Xilinx Kintex7芯片实现,可以实现10Gbps速率发送RDMA报文。同时,该平台可以支持与商业RDMA网卡互联。
	}
	\cventry{2015-3--2015-9}{南京大学}{}{南京}{}
	{
		\smallskip
		\emph{基于FPGA的软件定义中间件实现}
		\medskip \\
		软件 
	}
	\cventry{2013--2015}{南京大学}{}{南京}{}
	{
		\smallskip
		\emph{SimMIPS: 基于FPGA的MIPS嵌入式系统}
		\medskip \\
		SimMIPS是基于FPGA实现的片上系统,包括MIPS32核心,总线以及各种外设等。MIPS32核心按照标准MIPS32指令集实现,同时包括有内存管理单元(MMU)。使用C/C++开发的应用程序或操作系统,经过MIPS交叉编译工具编译后的代码可以直接在MIPS32核心中运行。
		关于本项目更多的细节可以访问 \url{https://github.com/jackyangNJ/SimMIPS}。
	}
	\cventry{2013--2014}{南京大学}{}{南京}{}
	{
		\smallskip
		\emph{SmartCar: 室内导航智能小车}
		\medskip  \\
		SmartCar旨在结合多种传感器技术,实现室内导航的功能。我们基于\href{http://zrobot.org/}{\emph{zrobot}} 平台,该平台由融合ARM核心的FPGA芯片控制, 使用C/C++, Java 和 Verilog HDL语言,结合To achieve autonomous navigation, we combined the traditional inertial navigation technology with image processing and added many sensors or devices to the car, such as acceleration sensor, gyroscope, USB camera, wireless network card, etc. For more information, you can visit \url{https://github.com/jackyangNJ/SmartCar}.
	}
	
	\cventry{2013 summer}{Institute of Computing Technology}{Chinese Academy of Sciences}{Beijing}{China}
	{
		\smallskip
		\emph{testing Xilinx soft processor MicroBlaze performance and building a prototype system}
		\medskip  \\
		The group I joined was to implement a system by using a new memory access protocol based on the asynchronous request and response message.
		This message-based memory could improve the effective utilization of memory bandwidth.
		My major work was to test the performance of soft processor MicroBlaze on target hardware to indicate whether it was suitable to be used in the system, and meanwhile I built a prototype system, a network composed of a interconnect 3x3 mesh of nine MicroBlaze processors on one FPGA chip.
	}
	
	
	% Publications from a BibTeX file without
	% multibib\renewcommand*{\bibliographyitemlabel}{\@biblabel{\arabic{enumiv}}}
	% for BibTeX numerical labels
	
	\nocite{*}
	% unsrt means the publications are displayed in the order in which
	% they appear in the publications.bib file
	\bibliographystyle{unsrt}
	\bibliography{publications}
	
	% Publications from a BibTeX file using the multibib package
	%\section{Publications}
	%\nocitebook{book1,book2}
	%\bibliographystylebook{plain}
	%\bibliographybook{publications}
	%\nocitemisc{misc1,misc2,misc3}
	%\bibliographystylemisc{plain}
	%\bibliographymisc{publications}
	
	
	% remove the comment-tag to diplay a signature under your cv
	\begin{comment}
	\begin{flushleft}
	\vspace{2cm}\hspace{3.5cm}
	Musterstadt, den \today
	\end{flushleft}
	
	
	\vspace{1cm} \hspace{3.5cm}
	% you could include a scanned signature here
	%\includegraphics[scale=0.75]{}\\
	Max Mustermann
	\end{comment}
	
	
	
\end{document}
