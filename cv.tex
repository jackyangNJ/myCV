% for further details and (very nice) examples see the official
% documentation at: http://www.ctan.org/pkg/moderncv
% the placeholder picture in this example is taken from the moderncv
% package.

\documentclass[11pt,a4paper]{moderncv}
\usepackage[ngerman]{babel}
\usepackage{verbatim}

% arguments: 'blue', 'orange', 'red', 'green', 'grey'
% and 'roman' (for roman fonts, instead of sans serif fonts)
\moderncvtheme[blue]{classic}

% character encoding
\usepackage[utf8]{inputenc}
\usepackage[T1]{fontenc}

% adjust page margins
\usepackage[scale=0.8]{geometry}
\recomputelengths

% adjust the column width
\sethintscolumnlength{3cm}

% personal data
\firstname{Jie}
\familyname{Yang}

% optional stuff, uncomment/remove these lines if not needed
\address{Nanjing University Xianlin Campus}{Nanjing}
\phone{+86 15365186708}
\email{jackyang7474@gmail.com}
%\photo[84pt]{picture/picture}

%\mobile{mobile}
%\title{Resume title}
%\fax{fax}
%\quote{Everything not saved will be lost. ---\textit{Nintendo Quit Screen}}

% uncomment to suppress automatic page numbering for CVs longer than one page
%\nopagenumbers{}


\begin{document}

% default section title for publications is 'Literatur'. we want
% to use 'Publikationen' instead.
\renewcommand{\refname}{Publikationen}
\maketitle

\section{Education}
\cventry{2010--2014}{B.Sc. Computer Science}{Nanjing University}{}{}{}
\section{Interests}
\cvline{}
{
    My major interests lie in computer architecture, Field Programmable Gate Array (FPGA) and embedded systems.
    During my undergraduate period, I was a lead programmer in some embedded systems.
    Unlike developing desktop software, there is more freedom to modify the hardware in embedded systems, so great efforts are needed to search for the best compromise between the hardware and the software.
    \newline
    I am also interested in programming on FPGA and embedded development boards, such as Altera DE2-70, STM32 and Raspberry PI.
    And I even have attempted to build a quadcopter myself from scratch.
}

\section{Scholarships and Awards}
\cvline{2011--2012}{National Endeavor Fellowship, The First Prize People Scholarship}
\cvline{2012--2013}{The Third Prize People Scholarship}

\section{Technical Skills}
\cvline{Languages}{C, C++, Java, Verilog HDL, Python, Perl, Matlab, \LaTeX}
\cvline{\small Operating Systems}{Windows, Linux}

\section{Project Experience}
\cventry{2012--present}{Nanjing University}{}{Nanjing}{China}
{
    \smallskip
    \emph{SimMIPS: a MIPS-based embedded system on FPGA}
    \medskip \\
    SimMIPS is an embedded system implemented on FPGA, including CPU, peripherals, multiple physical interfaces.
    As a simplified version of MIPS32 4Kc processor, SimMIPS covers most instructions of MIPS32 Release 1 Instruction Set, and programmes compiled by GNU GCC could be executed on our platform without no modification.
    SimMIPS would be adopted as an experimental platform in \emph{Comprehensive Experiments on Computer System} course for juniors in the Department of Computer Science in Nanjing University.
    You can visit \url{https://github.com/jackyang74/SimMIPS} for more details about our project.
}
\cventry{2013--2014}{Nanjing University}{}{Nanjing}{China}
{
    \smallskip
    \emph{SmartCar: an intelligent navigation car in the library}
    \medskip  \\
    SmartCar aims to navigate readers by driving to the right bookshelf autonomously in the library.
    We implemented the control system on the platform \href{http://zrobot.org/}{\emph{zrobot}}, using C, Java and Verilog languages.
    To achieve autonomous navigation, we combined the traditional inertial navigation technology with image processing and added many sensors or devices to the car, such as acceleration sensor, gyroscope, USB camera, wireless network card, etc. For more information, you can visit \url{https://github.com/jackyang74/SmartCar}.
}

\cventry{2013 summer}{Institute of Computing Technology}{Chinese Academy of Sciences}{Beijing}{China}
{
    \smallskip
    \emph{testing Xilinx soft processor MicroBlaze performance and building a prototype system}
    \medskip  \\
    The group I joined was to implement a system by using a new memory access protocol based on the asynchronous request and response message.
    This message-based memory could improve the effective utilization of memory bandwidth.
    My major work was to test the performance of soft processor MicroBlaze on target hardware to indicate whether it was suitable to be used in the system, and meanwhile I built a prototype system, a network composed of a interconnect 3x3 mesh of nine MicroBlaze processors on one FPGA chip.
}


% Publications from a BibTeX file without
% multibib\renewcommand*{\bibliographyitemlabel}{\@biblabel{\arabic{enumiv}}}
% for BibTeX numerical labels

\nocite{*}
% unsrt means the publications are displayed in the order in which
% they appear in the publications.bib file
\bibliographystyle{unsrt}
\bibliography{publications}

% Publications from a BibTeX file using the multibib package
%\section{Publications}
%\nocitebook{book1,book2}
%\bibliographystylebook{plain}
%\bibliographybook{publications}
%\nocitemisc{misc1,misc2,misc3}
%\bibliographystylemisc{plain}
%\bibliographymisc{publications}


% remove the comment-tag to diplay a signature under your cv
\begin{comment}
	\begin{flushleft}
	\vspace{2cm}\hspace{3.5cm}
	Musterstadt, den \today
	\end{flushleft}


	\vspace{1cm} \hspace{3.5cm}
	% you could include a scanned signature here
	%\includegraphics[scale=0.75]{}\\
	Max Mustermann
\end{comment}


\end{document}
